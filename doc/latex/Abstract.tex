%Abstract

Authors: Manuel Moser, Yannick Suter, Raffael Theiler\\
Title: Pedestrian dynamics in long, narrow hallways\\
\\
In this project, we want to have a closer look at pedestrians in narrow hallways, motivated by a situation at Zurich main station. To do this, we simulate pedestrians by agents with Matlab who walk according to some rules. We managed our agents to pass each other by, to look ahead a few meters and to decide where to walk next.\\
At first, we wanted to have a closer look at the pedestrian flux when the number of persons per minute entering is increased, when there are clearly more people moving in the same direction against a few in the other direction. Next, we wanted to analyze the influence of aggressive fast people in a rush, slowly moving obstacles and the influence of drunkard (randomized walking) on the pedestrian flux.\\
But soon, our attention turned more to building/creating everything on our own and less about a fast simulation of different situations.\\
The main outcome of our simulations was that pedestrians tend to get stuck or create jams as soon as there are lots of people trying to pass the same hallway. The exact same hallway can work smoothly if there are not too many people, but jams can arise quickly. And once a jam starts, it often spreads out because people have to stop and walk slower.\\
Finally, we arrived at the following conclusions: To improve the pedestrian flux, broadening a hallway is a superb solution. Therefore, at our situation scene at Zurich main station, it would be best if the hall users would try to leave more space for the pedestrians, and if the food shops Imagine and Nordsee wouldn't have chairs and tables outside.
