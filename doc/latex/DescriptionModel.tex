%DescriptionModel

\subsection{General considerations}
As we wanted to describe a situation with people, we chose a model based on discrete agents. They should be able to walk freely along their path until they hit an obstacle, in which case they should be able to determine a way to avoid the obstacle. Obstacles could be walls, objects placed in the path or simply other agents. The goal of any agent is to get to the other side of the path as quickly as possible. As the global situation changes with each "`step"' an agents takes and the appearance of new agents, a step-by-step iteration was chosen to propagate the situation in time in which the the optimal direction is determined all the time. The other approach of calculation a certain path from start to finish was rejected as it probably cannot be done for the uncertainties mentioned above, namely the random appearance of new agents which could block the calculated path.\\
Any agents goal is to get to the other side as quickly as possible, although our model cannot accommodate the requirement of the quickest possible path. Because of the step-by-step iteration, any situation is analyzed and the (hopefully) best way to go forward is chosen. As this is only a short time period, it cannot guarantee to give the best outcome overall. In short, we used a \textit{local search algorithm}.

\subsection{Walls and other static obstacles}
If a narrow hallway should be a narrow hallway, it needs two walls. Although this statement is obvious, it can be implemented in various ways. We chose to use static agents with a small radius to act as a wall, as it allows the creation of many different hallways. They can also be used as static objects representing obstacles like chairs and tables which could stand in the path an agent has to go to get to the other side. In our way, this was a very flexible way which also allows a quick adapted to other situations if necessary.

\subsection{Agents}
The basis of the simulation is the agent. An agent should represent a person in real life. We assumed that the hallway was not a place to linger about, therefore they should try to get to the other side as quickly as possible. In order to do so, they need to be aware of all the things around them that they might bump into. This was the origin of the thought that any agents only looks forward as the obstacles will be in the path before them. In our model, they also need some intelligence to get around an obstacle and avoid running into other agents. To do so, the agent should consider all things within a circle of defined radius in front of him while everything else doesn't bother him. Again, this tries to get a local solution to our problem in the hope that the overall solution is still a good one. As one usually doesn't walk backwards, our agents only walk forewards. This might not be true for all situations but simplifies the model considerably.\\

\noi In comparison to previous models which used a force field to guide the agents, we thought that this approach is more realistic as the force field approach given that the force field already defines the optimal solution one is looking for. It should also prove to be more adaptable within a reasonable time frame to other problems.


