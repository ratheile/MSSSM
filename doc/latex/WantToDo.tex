%Want-To-Do List 
% To be ereased when finishing up project!

\section{Want-To-Do-List}

\begin{itemize}

\item im latex-folder inne euse report zu "report" umbenenne (statt wie bisher reportTemplate)
\item Titelblatt: Titel so guet oder  i zwei Zeile ufteile?
\item Titelblatt: nachli meh Platz ief�ege zw�sched die Sache (zw�sched ETH-Bildli, Lecture.., Titel, und Zurich December 2012)?
\item Titelblatt: December 2012 guet?
\item chamer ae oe ue au na enable? ;)
\\

Im File reportTemplate.tex :
\item was bi date ietueh? today, oder doch eifach scho direkt de 14.12. wemmers m�end abgeh? oder chamer en art "min(today, 14.12.2012) mache" ? :P Uf di ander Siite griifemer eh niened uf das Datum zrugg, uf de Frontpage staht December 2012 (Vorlag isch gsi May 2008)...
\item wieso hets nach inputcover,newpage sonen fette \%\%\%-Balke, denn weder newpage und denn s agreement? Chamer da nid de Balke + einisch s newpage usel�sche?
\item allgemein: reportTemplate versch�nern/verk�rzen!
\item: Ich w�rde die einzelnen Titel und jeweiligen newpages ins jeweilige dokument nehmen (wie jetzt bei der Wanttodolist getan). Warum (nicht) ?
\item Agreement for free-download als separates File - gut so?
\item Agreement for free download: 3. Name einf�gen; versch�nern
\item "fixes" pdf is LaTex ibinde m�glich? (und wie zellts denn d Siite?): well mer m�end ja na die Declaration ietueh, da ischsi offiziell: \\ $http://www.ethz.ch/faculty/exams/plagiarism/confirmation_en.pdf$ (eifach vo dem Link wos drin gha hend) ; s isch allerdings nur f�r 1 Person, ich chum ned ganz drus �b mer das ding eifach 3mal m�end usf�lle (jede einzeln) oder �bs da na en gruppe-version g�bti, well online isch n�t... fallsd d antwort weisch: gib sie mir ; und sust schrib ich mal dere kontaktperson obe rechts.

\item Theiler: was wie womit ignoriere?

\item Fuer github: git config --global user.name "sutery"\\
git config --global user.email "sutery@student.ethz.ch"

\end{itemize}

\newpage