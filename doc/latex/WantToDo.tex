%Want-To-Do List 
% To be ereased when finishing up project!

\section{Want-To-Do-List}

\begin{itemize}

\item[!] GENERAL :
\item WIE/WO Daten rausziehen aus Agents?
\item gestaute (noch unfertig) agents nch simulationsende einbeziehen oder nicht, f�r welche Zwecke?
\item Densityup/Densitydown einbinden (hoffentlich noch m�glich?)
\item Agent seine Geburtsiteration mitgeben, und bei Auswertung auch Todesiteration, um Zeit auszuwerten?
\\

\item[!] OUTPUT :
\item Untersuche Situationen wenn: Anz. Agents variiert, Gangbreite variiert.
\item Plot mit: x-Achse Zeit, y-Achse Personen pro Zeitabschnitt (evtl. mehrere deltaT) erzeugt/gel�scht, je oben&unten, f�r verschiedene Gangbreiten
\item Wegl�ngen? (Problem: abh�ngig von Anz. Agents im System, und diese �ndert sich immer wieder
\item Jeder Agent: gesamthaft gelaufene Strecke pro deltaT
\item Wie viele Agents im System drin?
\\

\item[!] TEST:
\item Mit Dispersionfactor rumspielen
\item Mal Standardabweichung der Geschweindigkeit etwas erh�hen
\\

\item[!] SIMPLE:
\item Bei den Constants bei Speed und Radius hinschreiben, dass 3*Standardabweichung kleiner als der Mittelwert sein muss.
\\

\item[!] DOKUMENTATION :
\item F�r alle Funktionen einen Kurzbeschrieb erstellen mit: Sinn und Zweck, Rechenmethode, evtl noch Kommentare dazu.
\item Code auskommentieren!
\item Kapitel "How to start our simulation" und was man wie w�hlen kann, ebenfalls run auskommentieren
\item Kapitel Limitations of our model (dort simplifications reinnehmen, 180�-limitation verhindert "ausbrechen" (Bild), etc.
\\

\item[!] INFO :
\item \textbf{sim.spawned}: wieviele werden gespawnt in jedem Iterationsschritt. 1.Zeile = oben; 2.Zeile = unten
\item \textbf{sim.result}: wieviele kommen an in jedem Iterationsschritt. 1.Zeile=oben, 2.zeile = unten.
\item \textbf{sim.additionalReport} 1.Zeile: Von allen Agents in diesem Iterationsschritt zur�ckgelegte Strecke , 2.Zeile  Agents im System in jedem Iterationsschritt
\item \textbf{fehlt noch}: Funktion evaluateAgent welche beim "Tod" eines Agents seine Daten (gelaufene Wegl�nge, evtl noch Geburts- und Todesiteration) ausliest und in eine sim.... matrix speichert!

\end{itemize}

\newpage