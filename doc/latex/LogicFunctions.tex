%LogicFunctions.tex

\subsubsection{General considerations}
The heart of the simulation is the function \textit{logicFunction.m}, which determines the path any agent will choose to get to the other side of the hallway. To be more precise, it determines only the next step an agent will take and not the whole path. It relies heavily on the two functions \textit{xValuesLogic.m} to deal with other agents and \textit{xWallLogic.m} to deal with agents representing the wall or static obstacles. At first, the functioning of \textit{xValuesLogic.m} will be explained and afterwards the functioning of \textit{xWallLogic}.\\

\subsubsection{How to get $\beta_\y{Links}$ and $\beta_\y{Rechts}$?}
\begin{figure}[h!]
	\centering
		\includegraphics[width=0.80\textwidth]{pictures/beta.PNG}
	\caption{The graph shows the angles and variables used to get $\beta_\y{Links}$ and $\beta_\y{Rechts}$. $\alpha_X$ is the angle between the two agents with respect to the $y$-axis. This depiction was engineered to work also for agents walking the other way.}
	\label{fig:beta}
\end{figure}

\noi For our model, it is crucial to determine where an agent shouldn't go. The function \textit{getBeta.m} returns the angles which describe the interval leading to a collision. A graphical depiction of the situation is given in figure \ref{fig:beta}. The equations (\ref{logik3}) to (\ref{logik5}) were used to get $\beta_\y{Links}$ and $\beta_\y{Rechts}$. They had to be converted into the angles given with respect to $\varphi$, $\beta_{\varphi,\y{ left}}$ and $\beta_{\varphi,\y{ right}}$ as shown in equations (\ref{logik6}) to (\ref{logik7}).
\begin{equation}\label{logik3}
	\gamma = arccos\brac{\frac{r_S}{d}},\ \alpha = arctan\brac{\frac{\Delta y}{\Delta x}}
\end{equation}
\begin{equation}\label{logik4}
	\beta_\y{Links} = \gamma + \alpha - \frac{\pi}{2}
\end{equation}
\begin{equation}\label{logik5}
	\beta_\y{Rechts} = + \alpha + \frac{\pi}{2} - \gamma
\end{equation}
\begin{equation}\label{logik6}
  \beta_{\varphi,\y{ left}} = \frac{\pi}{2} - \beta_\y{Rechts} = \pi - (\gamma + \alpha)
\end{equation}
\begin{equation}\label{logik7}
  \beta_{\varphi,\y{ right}} = \frac{\pi}{2} - \beta_\y{Rechts} = \gamma - \alpha)
\end{equation}
\noi This works between agents as well as between agents and the wall agents. Care was taken to engineer a calculation that allows for it to be used for agents walking in both directions.

\subsubsection{xValuesLogic.m}
\text{xValuesLogic.m} distinguishes three different cases.
\begin{itemize}
	\item For two crossing agents or if the agent in front of the agent in question is slower, we used equation (\ref{logik1}) to get $x_\y{out}'$. It was also used for two not moving agents, setting $\Delta v$ equal to an arbitrary value given in \textit{STANDOFF}.
	\begin{equation}\label{logik1}
		x_\y{out}' = \frac{1}{\dis (|x - \alpha_X|)^{\brac{\frac{-\Delta v}{a}}}} = (|x - \alpha_X|)^{\brac{\frac{\Delta v}{a}}},\ \Delta v < 0
	\end{equation}
	\noi All values which correspond to a collision course in $x_y{out}'$ are set to zero. This also deals with the singularity of equation (\ref{logik1}) as it is set to zero. This done using the $\beta$-angles shown before. Afterwards, $x_\y{out}'$ is normalized and modificated further using equation (\ref{logik2}).
	\begin{equation}\label{logik2}
		x_\y{out} = x_\y{out}' \cdot \frac{b}{max(x_\y{out}')} \cdot \brac{\frac{r_S}{d}}^c
	\end{equation}
	\noi The variables $a$ (called \textit{SLOPEFACTOR}), $b$ (\textit{HEIGHT}) and $c$ (\textit{REPULSIONAGENT}) have to chosen in a way that the simulation runs smoothly. The term $\frac{b}{max(x_\y{out}')}$ normalized the function to a maximum value $b$ while the term $\brac{\frac{r_S}{d}}^c$ controls that the repulsive influence gets stronger the closer the two agents get. $c$ is usually chosen to be larger than 1.\\
	
	\noi If the $x_\y{out}$ given in equation (\ref{logik2}) would be returned, the agent in question would aim to miss the other agent exactly. We thought that this would be to close as in realität, one also leaves a bit of space if possible between each other. Therefore we introduced an offset given as \textit{WALLANGLEOFFSET} which gives the angle additionally to the $\beta$ angles for which an agent should aim to. To account for this, $x_\y{out}$ is modified with an linear interpolation between the values at $\beta + WALLANGLEOFFSET$ and $\beta$ (which was set to zero before).

	\item If the agent in front of the agent in question is faster, a gaussian curve was used with the mean $\alpha_X$ and standard deviation $rS/d$. It is then modified further with $\Delta v$ and \textit{HEIGHT} to make it a weak influence.
	\item For two agents moving with the same speed, the influence is set two zero by returning a vector of zeros.
\end{itemize}


\subsubsection{xWallLogic.m}
To avoid hitting the wall, we used a very simple approach. Every angle corresponding to a collision course is set to a negative value accoring to equation (\ref{logik8}).
\begin{equation}\label{logik8}
	x_\y{Out} = x \cdot \frac{a}{d - rS}
\end{equation}
\noi As before for agents, an offset is introduced so the agent in question doesn't just try to avoid the wall-agent but also to leave some buffer space. The offset is also given in \textit{WALLANGLEOFFSET}, $a$ can be accessed with the constant variable \textit{WALLFACTOR}. $a$ has to be set negative as otherwise the wall would have an attractive force. To set a good value for this factor $a$ is quite delicate because if it is too low agents will be stuck in the wall while if it too high they will never approach the wall even slightly. 



