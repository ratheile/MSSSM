% Summary and Outlook

First of all, we can say that our simulation series showed that our own model works well. It may have its limitations we're well aware of, but it runs properly what it's supposed to do.
A very nice point about our simulation is that it can be applied onto other problems quickly as for example by adding walls or specifying agents.\\

\noi If one would want to go further with this project, there are some points clear enough where to start, but hard to do: It's to un-simplify our model. For example, insted of only walking forwards with an almost 180� vision could be changed to 360� of possibilities, but this would also require much more sophisticated path finding algorithms and couldn't be done that fast.\\
One thing we did not bear in mind until we were in a very final state of the coding process was the performance of our simulation. First tests showed that looping trough arrays, even if they are only a few 1000 entries large, is very slow. Following this, we could speed up the drawing part of our simulation by almost 200\%. If one wants to simulate bigger environments with a huge number of agents and wall points, a faster approach of the Iteration implementation should be taken into consideration.\\
Another point one might be interested in would be to embed groups into the model as in reality, groups of people flocking together are a common view. This would imply some kind of sticking-together algorithm such that the groups wouldn't be torn apart.\\
Another challenge would be to try to improve our weighing function. As seen in chapter $hier Verweis zur influence-Sphere-Diskussion$, our weighing function is a bit too focussed on agents very close to other agents and thinks less about agents in some distance.\\
In comparison to those improvements, adjusting the parameters to work better as an improvement almost seems easy. But also there, just the question "what is better and why?" can often not be answered clearly.\\

\noi As for our fundamental research questions, we could investigate most of what was our interest. We showed in $hier verweis zur width-diskussion$, that wider hallways lead to better flux, if there are lots of jams. This can be seen in reality as for the Christmas market, Nordsee and Imagine put away their tables and chairs which leads to a good passenger flux even if there are lots of people.\\
We did not do lots of our originally planned simulations on specified agents like big/small, fast/slow agents because it turned out to be far more interesting to work with Gaussian distributed agents while investigating the dependency between the passenger flux and other influences as the hallway's width, the flux density and so on.\\
Our own kind of intelligence, the agent being able to look ahead, worked out quite nicely. If there are not too many other agents, they'll find a way to transit the hallway without crashing, which is mostly what a commuter's thoughts are like.\\
To sum up, we mostly found what we were looking for: A wider hallway leads to better pedestrian density, and egoists trying to overtake everybody can cause jams. In reality, we can't do much about all this, but it was nice having our thoughts confirmed.