%Experiments

\noi Listed below are the carried out simulations with their parameters. For each series of simulations, a brief explanation is given to state the questions which will be looked at with the actual simulation series.\\
All parameters can be found in the corresponding logfiles. Usually only one parameter was varied while all others were kept constant. We chose a mean radius for the agents of 0.25 meters with a standard deviation of 0.03 meters. A mean velocity of 1.5 meters per second was chosen with a rather large standard deviation of 0.25 meters per second.\\
It should be noted that we usually used high people flux densities since we wanted to test the model under stress conditions. Therefore we expected a considerable amount of failure in the examined situations.

\subsection{Influence of different pedestrian flux densities}
To check the influence of different densities on our model, we ran the simulation with the density combinations 0.4/0.4, 0.4/0.6, 0.4/0.8, 0.4/1.0, 0.6/0.6, 0.6/0.8, 0.6/1.0, 0.8/0.8, 0.8/1.0 and 1.0/1.0. The first number represents the value chosen for \texttt{DENSITYDOWN}, the second for \texttt{DENSITYUP}. We didn't run the inverted combinations due to the situations symmetry. The simulations were repeated with three different seeds each, 51, 71 and 91. A high value for \texttt{DISPERSIONFACTOR} of 1.0 was chosen which corresponds to people having a strong tendency to try to overtake slow agents.

\subsection{Influence of overtaking or lane formation on the success of the model}
It soon became clear to us that the parameter \texttt{DISPERSIONFACTOR} would be absolutely crucial if one wants to force the model to succeed. A negative value encourages the agents to form lanes while a positive value encourages them to try finding their own way. In order to investigate this property, specially with the dilemma of personal vs. group success in mind, we ran a simulation series where we incremented the \texttt{DISPERSIONFACTOR} from -0.2 to 1 each time by 0.1. A high density flux of 1 person per second on both sides was used to test the model in a stress situation. This was done for three different seeds each, 51, 151 and 351. 

\subsection{Influence of the radius of sight of an agent}
The constant variable \texttt{INFLUENCESPHERE} determines the radius of the semi-circle in which the agent considers other agents around him. With flux densities of 1.0 each and a \texttt{DISPERSIONFACTOR} of 0.7, the \texttt{INFLUENCESPHERE} was tested using the values 1.5, 2.0, 2.5 and 3.0 (in meters). This was done for three seeds each, 51, 77 and 151.

\subsection{Influence of the hallway width on the success of the simulation}
To account for the influence of the width of the hallway on the success of the simulation, we did a simulation series with different widths. The tested widths were 2.2, 2.5, 2.8, 3 and 3.5 meters. A high density flux of 1 person per second on both sides was used with a \texttt{DISPERSIONFACTOR} of 0.75 corresponding to a high number of overtaking attempts. The simulations were repeated with the seeds 51, 77 and 151 each.

\subsection{Simulating measurements of the main station Zurich}
Saturday, Nov 17th, we did some quick measurements right at Zurich main station to have some data we could try to compare. Two measurements were taken, only some minutes lay between these, that was when we measured the length and breadth of our corridor. The measurements were:
\begin{enumerate}
\item The "boring" measurement: During 2 minutes, 14 pedestrians headed towards tracks 3-18, and 20 pedestrians directed towards tram station "Bahnhofsquai". No problems at all, very fluently.
\item The "crowded" measurement: During 2 minutes, 41 pedestrians headed towards tracks 3-18, and 33 pedestrians directed towards tram station "Bahnhofsquai". People got stuck, ran into each other, had to walk stop-and-go-like for some moments.
\end{enumerate}

%Hier noch weiterschreiben
